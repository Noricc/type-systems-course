\documentclass[presentation]{beamer}

\usepackage[utf8]{inputenc}
\usepackage[T1]{fontenc}
\usepackage{tikz}

\usetikzlibrary{arrows.meta}

\usepackage[sfdefault,book]{FiraSans} %% option 'sfdefault' activates Fira Sans as the default text font
\usepackage{FiraMono}

\renewcommand*\oldstylenums[1]{{\firaoldstyle #1}}

\usepackage{color}
\definecolor{light}{rgb}{0.5, 0.5, 0.5}
\def\light#1{{\color{light}#1}}

\author{Noric Couderc}
\date{\today}
\title{Case Study: Imperative Objects}

\begin{document}


\maketitle

\section{Intro}
\begin{frame}
  \begin{itemize}
  \item What is object oriented programming?
  \item How to implement OOP in our language?
  \end{itemize}
\end{frame}

\section{Object Oriented Programming}

\section{Objects}
\begin{frame}{Five rules of object-oriented programming}
\begin{itemize}
\item Multiple Representations
\item Encapsulation
\item Subtyping
\item Inheritance
\item Open recursion
\end{itemize}
\end{frame}

\begin{frame}[fragile]{Multiple representations}
\begin{block}{Dynamic dispatch}
\begin{verbatim}
walks(dog)
\end{verbatim}
\begin{verbatim}
walks(cat)
\end{verbatim}
\end{block}

\begin{block}{Syntactic sugar}
\begin{verbatim}
barks(dog, cat) = dog.barks(cat)
\end{verbatim}
\end{block}
\end{frame}

\begin{frame}{Encapsulation}
  \centering
  Objects have an \textbf{inside} and an \textbf{outside}\\
  If you're \textbf{outside} you can't see \textbf{inside}
\end{frame}

\tikzset{
  classbox/.style={
    rectangle,
    draw,
    text width=1.5cm,
    text centered,
    },
}

\begin{frame}{Subtyping}
  \centering
  \begin{tikzpicture}[thick,>=Stealth]
    \node[classbox] (interface) at (0,0) {Animal};
    \node[classbox] (class1) at (-1, -2) {Cat};
    \node[classbox] (class2) at (1, -2) {Dog};

    \draw[->] (class1) -- (interface);
    \draw[->] (class2) -- (interface);
  \end{tikzpicture}
\end{frame}

\begin{frame}{Inheritance}
  \centering
    \begin{tikzpicture}[thick,>=Stealth,border=1cm]
      \node[classbox] (interface) at (0,0) {\textbf{Animal}\\walk};
      \node[classbox] (class1) at (-1, -2) {\textbf{Cat}\\meow\\\light{walk}};
      \node[classbox] (class2) at (1, -2) {\textbf{Dog}\\bark\\\light{walk}};

      \draw[->] (class1) -- (interface);
      \draw[->] (class2) -- (interface);
    \end{tikzpicture}
\end{frame}

\begin{frame}{Open recursion}
  \centering
  Methods of an object can call each other
\end{frame}

\section{Classes}

\section{Open Recursion}

\end{document}
